\section{Simulation}

In order to use the previously proposed models, pictures of tomato plants are required. Since we do not have a greenhouse in which we can take the pictures, a simulation will be used instead. Tomato plants grow according to a pattern. The leaves have a certain shape and the flowers have a certain amount of petals in a certain color. But still each plant is individual, as parameters like the number and position of branches can vary. The simulation should be able to automatically generate a multitude of tomato plants varying in these parameter. Each parameter has a certain range in which it can be selected.

Since we decided to focus the project on the prediction of the amount of leaves on the plant, the simulation of diseases and ripeness becomes secondary. While, we will still explain how these parts could be simulated in general, we will concentrate on the simulation of different amounts of leaves on the plants.  