\graphicspath{{members/ssr/figures/}}

\subsection{Assumptions and High-Level Design}
\input{members/ssr/authors}

The designed system for the purpose of yield estimation is founded on one central assumption which is provided 
by the user-view:

\begin{center}
    \textit{The canopy size or green leaf area is positively correlated with yield.}    
\end{center}

This also means that damaged leaves don't contribute to the yield due to damaged bio-active tissue.\\

For this purpose the central metric for yield estimation is the \textit{leaf area index} (LAI).
This is achieved by combining two computation pipelines - each estimating specific metrics - for the final estimation.

\begin{itemize}
    \item \textbf{Pipeline A:} Estimates the \textit{leaf area} (not LAI) which is visible from above of the plant
    taken from a camera.
    \item \textbf{Pipeline B:} Estimates the plant height based on an image taken from a different camera
    with a specific setup.
\end{itemize}

Both metrics are then combined with additional statistical corrections to estimate the actual LAI.
Again, the LAI is different from the \textit{leaf area} in the way that it contains
the \textit{entire} existing leaf area and not only the leaf area which is visible from above.
