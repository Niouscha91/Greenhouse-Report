\graphicspath{{members/ssr/figures/}}

\subsection{Integration with Simulation}\label{subsec:integration-with-simulation}

The way in which this project was conducted led to some limitations in the way the simulation was executed
which have to be explained in order to understand how it affected the modelling and the entire project.\\

Normally, the modelling must be done (or progressed quite far) before the requirements for the simulation are known.
The reason is that the model must first be formally defined with all of its parameters and then the evaluation can start
on how to simulate the model - either by using existing tools or by creating them.\\

In order to fulfill the time constraints of this project however the modelling and simulation team worked in parallel and
this affected choices made by the modelling group in order to archive a compromise between \textit{preferred} and \textit{practical} design choices.

The reasons are that the relevant metrics and data gathering setup wasn't yet known until later in the modelling
since many approaches had to explored until one promissing design was found.
The limitations mainly concern the simulation of an accurate plant's appearance and the physical material properties
of the plant's surface.\\

Following metrics has been selected for the simulation and can be provided with data (or need yet real-world probability distributions):\\

\begin{center}
    \begin{tabular}{ |l|l| }
        \hline
        $\theta_{col,leaf}$ & Material color distributions \\
        $\theta_{col,earth}$ & Earth color distributions  \\
        $\theta_{tilt}$ & Leaf tilt distribution  \\
        $\theta_{height}$ & Plant height  \\
        $\theta_{d,level}$ & Level distances  \\
        $h_{leaf}(level)$ & Leaf count distribution per level (or per-level leaf density) \\
        \hline
    \end{tabular}
\end{center}